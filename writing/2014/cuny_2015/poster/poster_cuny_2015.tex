\documentclass[a0,portrait]{a0poster}
\pagestyle{empty}
\setcounter{secnumdepth}{0}

\usepackage[absolute]{textpos}
%\usepackage[colorgrid,texcoord, gridunit=in]{eso-pic}
\usepackage{graphicx}

\usepackage[T1]{fontenc}
\usepackage[latin1]{inputenc}
\usepackage{ae}
\usepackage{palatino}
\usepackage{mathpazo}
\usepackage{multirow}
\usepackage{amsmath, amsfonts, amssymb}
\usepackage{caption}
\captionsetup[figure]{labelformat=empty,position=top,skip=0cm}

\usepackage[notocbib]{apacite}

\usepackage{wrapfig,times}
\usepackage{rotating}

%\usepackage{gb4e}
\usepackage{texpower}
\usepackage{rotating}
\usepackage{booktabs}
\usepackage{multirow}


% These colours are tried and tested for titles and headers. Don't
% over use color!
%\usepackage[dvipsnames,usenames]{color}
\usepackage{color}
\definecolor{Red}{rgb}{0.9,0.0,0.1}
\definecolor{myBlue}{RGB}{0,0,255}
\definecolor{lightblue}{RGB}{160,188,250}
\definecolor{myGreen}{RGB}{50,205,50}
\definecolor{MyGray}{RGB}{109,117,123}
\definecolor{lightgray}{RGB}{194,208,218}
\definecolor{redgray}{RGB}{232,208,218}
\definecolor{weakpink}{RGB}{222,89,234}
\definecolor{strongyellow}{RGB}{243,225,27}

\definecolor{examplecolor}{RGB}{175,175,175}
%\definecolor{superlightgray}{RGB}{215,215,255}
\definecolor{superlightgray}{RGB}{205,215,255}
\definecolor{xxlightgray}{RGB}{225,235,255}


\newcommand{\red}[1]{\textcolor{Red}{#1}}
\newcommand{\blue}[1]{\textcolor{myBlue}{#1}}
\newcommand{\lightblue}[1]{\textcolor{lightblue}{#1}}
\newcommand{\green}[1]{\textcolor{myGreen}{#1}}
\newcommand{\gray}[1]{\textcolor{MyGray}{#1}}
\newcommand{\lightgray}[1]{\textcolor{lightgray}{#1}}
\newcommand{\redgray}[1]{\textcolor{redgray}{#1}}
\newcommand{\white}[1]{\textcolor{white}{#1}}
\newcommand{\superlightgray}[1]{\textcolor{superlightgray}{#1}}
\newcommand{\xxlightgray}[1]{\textcolor{xxlightgray}{#1}}
\newcommand{\weakpink}[1]{\textcolor{weakpink}{#1}}
\newcommand{\strongyellow}[1]{\textcolor{strongyellow}{#1}}

% see documentation for a0poster class for the size options here
\let\Textsize\large
\def\Head#1{\noindent\hbox to \hsize{\hfil{\large\color{black} #1}}\bigskip}
\def\LHead#1{\noindent{\LARGE\color{myBlue} #1}}
\def\Subhead#1{\noindent{\large\color{black} #1}}
\def\Title#1{\noindent{\VeryHuge\color{Red} #1}}

 \catcode`^=7
 \catcode`_=8

\TPGrid[100mm,130mm]{15}{25}  

\parindent=0pt

\parskip=0.5\baselineskip



 \TPMargin*{0.2\TPHorizModule}
\setlength{\TPboxrulesize}{3pt}

\begin{document}



%%%%%%%%%%%%%%%%%%%%%%%%%%%%%%%%%%%%%%%
% TITLE STUFF
%%%%%%%%%%%%%%%%%%%%%%%%%%%%%%%%%%%%%%%

\begin{textblock}{1.5}(0.1,0.4)
\includegraphics[width=\textwidth]{pics/stanfordlogo.png}
\end{textblock}

\begin{textblock}{12}(1.5,0)
\begin{center}
\baselineskip=3\baselineskip \Title{Non-sinking marbles are wonky: world knowledge in scalar implicature}
\end{center}
\end{textblock}

\begin{textblock}{1.3}(13.6,0.3)
\vspace{1em}
\includegraphics[width=\textwidth]{pics/cocologo.png}
\end{textblock}

\begin{textblock}{15}(0,2)
\LARGE{ \textbf{Judith Degen}  \texttt{<jdegen@stanford.edu>}  \hspace{3cm}  \textbf{Noah D.~Goodman} \texttt{<ngoodman@stanford.edu>}}
\end{textblock}

\begin{textblock}{15}(0,2.2)
\begin{center}
\Large{Department of Psychology, Stanford University}
\end{center}
\end{textblock}


%%%%%%%%%%%%%%%%%%%%%%%%%%%%%%%%%%%%%%%
% ABSTRACT
%%%%%%%%%%%%%%%%%%%%%%%%%%%%%%%%%%%%%%%

\begin{textblock}{7.2}(-0.1,2.6)
  \begin{center}
   \mbox{\colorbox{superlightgray}{
         \begin{minipage}{1.0\textwidth}
         \superlightgray{--------------------------------------------------------------------------}
	\vspace{10.4cm}
         \end{minipage}
      }
   }
\end{center}

\end{textblock} 

\begin{textblock}{7}(0,3)
  \LHead{Abstract}

\Large
Listeners We present a model of pragmatic reasoning within the Rational Speech Acts framework.
\end{textblock}



%%%%%%%%%%%%%%%%%%%%%%%%%%%%%%%%%%%%%%%
% INTRO
%%%%%%%%%%%%%%%%%%%%%%%%%%%%%%%%%%%%%%%
\begin{textblock}{7.2}(-0.1,5.4)
  \begin{center}
   \mbox{\colorbox{xxlightgray}{
         \begin{minipage}{1.0\textwidth}
         \xxlightgray{--------------------------------------------------------------------------}
	\vspace{27.05cm}
         \end{minipage}
      }
   }
\end{center}

\end{textblock} 


\begin{textblock}{7}(0,5.8)
  \LHead{Introduction - syntactic alternations\dots}
\large

\begin{textblock}{6.8}(0.1,6)
  \begin{center}
   \mbox{\colorbox{white}{
         \begin{minipage}{1.0\textwidth}
         \white{--------------------------------------------------------------------------}
	\vspace{15cm}
         \end{minipage}
      }
   }
\end{center}
\end{textblock} 


\begin{textblock}{6.4}(0.2,6.4)

\large 

\red{\textbf{In psycholinguistics: a window onto production pressures}}

Extensive study of the choice between syntactic forms assumed to be  \red{meaning-equivalent}, e.g., \emph{that}-omission \normalsize \cite{ferreira2000}, \large active/passive \normalsize \cite{bock1980}, \large ditransitive \normalsize \shortcite{bresnan2007} \large

\textbf{Availability-based production:}\\
Avoid suspension of speech: utter available material first

\textbf{Robust communication: Uniform Information Density}\\
Within the bounds defined by grammar, produce utterances that distribute information uniformly across the \\ linguistic signal \normalsize \cite{jaeger2010, levy2007} \large
\end{textblock}

\begin{textblock}{2.5}(4.9,9.55)
  \mbox{\colorbox{black}{
         \begin{minipage}{0.75\textwidth}
   	\vspace{10pt}         
	\white{Info($u$) = -log$p(u)$}
	\vspace{0.01pt}         
	\end{minipage}
	}}	
\end{textblock}

%%%%%%%%%

\begin{textblock}{6.8}(0.1,10.4)
  \begin{center}
   \mbox{\colorbox{white}{
         \begin{minipage}{1.0\textwidth}
         \white{--------------------------------------------------------------------------}
	\vspace{7.7cm}
         \end{minipage}
      }
   }
\end{center}
\end{textblock} 


\begin{textblock}{6.4}(0.2,10.75)

\large 

\red{\textbf{In theoretical linguistics: a window onto meaning differences}}

Choice between syntactic forms assumed to be driven by \red{meaning differences} \normalsize \shortcite{dor2005,gropen1989,pinker1989} \large

``a difference in syntactic form always spells a difference in meaning''  \normalsize \cite{bolinger1968} \large



\end{textblock}



%%%%%%%%%%%%%%%%%%%%%%%%%%%
% TEST CASE
%%%%%%%%%%%%%%%%%%%%%%%%%%%

\begin{textblock}{7.54}(7.4,2.6)
  \begin{center}
   \mbox{\colorbox{xxlightgray}{
         \begin{minipage}{1.0\textwidth}
         \xxlightgray{--------------------------------------------------------------------------}
	\vspace{7.9cm}
         \end{minipage}
      }
   }
\end{center}
\end{textblock} 

\begin{textblock}{7.5}(7.6,3)
\LHead{The test case: simple vs.~partitive \emph{some}}

\begin{textblock}{7.14}(7.6,3.2)
  \begin{center}
   \mbox{\colorbox{white}{
         \begin{minipage}{1.0\textwidth}
         \white{--------------------------------------------------------------------------}
	\vspace{5.1cm}
         \end{minipage}
      }
   }
\end{center}
\end{textblock} 

\begin{textblock}{1.1}(8.6,3)
  \begin{center}
   \mbox{\colorbox{lightgray}{
         \begin{minipage}{1.0\textwidth}
         \lightgray{-------}
	\vspace{4cm}
         \end{minipage}
      }
   }
\end{center}
\end{textblock}

\begin{textblock}{7.5}(7.7,3.3)

Alex ate \hspace{0.1cm} \red{some} \hspace{2.7cm} cashews. \hspace{5pt} 	\textbf{[simple \emph{some}; shorter form]}\\
Alex ate \hspace{0.1cm} \red{some of the} \hspace{0.05cm} cashews. \hspace{5pt}  \textbf{[partitive \emph{some}; longer form]}

Alex ate \hspace{0.1cm} \normalsize \red{SOME} \hspace{2.7cm} \large  cashews. \hspace{5pt} \textbf{[combined meaning contribution]}

\end{textblock}

\end{textblock}

\end{textblock}

%%%%%%%%%%%%%%%%%%%%%%%%%%%%%%%%%%%%%%%
% QUESTIONS
%%%%%%%%%%%%%%%%%%%%%%%%%%%%%%%%%%%%%%%

\begin{textblock}{7.54}(7.4,4.7)
  \begin{center}
   \mbox{\colorbox{superlightgray}{
         \begin{minipage}{1.0\textwidth}
         \superlightgray{--------------------------------------------------------------------------}
	\vspace{10cm}
         \end{minipage}
      }
   }
\end{center}

\end{textblock} 

\begin{textblock}{7.1}(7.5,5.1)
  \LHead{Hypotheses}

  \large

\begin{enumerate}
	\item \textbf{Meaning and production pressures operate in parallel in quasi-alternations} (when meanings of two forms are \emph{similar enough}).
	\item \textbf{Gradient Alternation Hypothesis:} Effects of production pressures are more pronounced, the more similar the meanings that the speaker intends to convey by using one of the two forms are.
\end{enumerate}
	
\end{textblock}


%%%%%%%%%%%%%%%%%%%%%%%%%%%%%%%%%%%%%%%%
%% TESTING HYPOTHESIS 1
%%%%%%%%%%%%%%%%%%%%%%%%%%%%%%%%%%%%%%%%

\begin{textblock}{7.54}(7.4,7.7)
  \begin{center}
   \mbox{\colorbox{xxlightgray}{
         \begin{minipage}{1.0\textwidth}
         \xxlightgray{--------------------------------------------------------------------------}
	\vspace{36cm}
         \end{minipage}
      }
   }
\end{center}

\end{textblock} 
\begin{textblock}{7.5}(7.5,8.1)
  \LHead{Study 1: parallel pressures}
  
  \large

\begin{textblock}{7.14}(7.6,8.3)
  \begin{center}
   \mbox{\colorbox{white}{
         \begin{minipage}{1.0\textwidth}
         \white{--------------------------------------------------------------------------}
	\vspace{4.2cm}
         \end{minipage}
      }
   }
\end{center}
\end{textblock} 

\begin{textblock}{7.1}(7.7,8.7)

\textbf{Dataset} 1237 cases of \emph{some}-NPs (269 partitives, 23\%) from Switchboard corpus after excluding cases that can only occur in one of the two forms (pronouns, singular count nouns, idioms)

\end{textblock}

\begin{textblock}{7.14}(7.6,9.8)
  \begin{center}
   \mbox{\colorbox{white}{
         \begin{minipage}{1.0\textwidth}
         \white{--------------------------------------------------------------------------}
	\vspace{14cm}
         \end{minipage}
      }
   }
\end{center}
\end{textblock}

\begin{textblock}{7.1}(7.7,10.3)

\textbf{Predictors} entered in mixed-effects logit model predicting partitive use:

\vspace{.5cm}

\begin{tabular}{p{10.5cm} l l}
\toprule
\centering \textbf{Meaning} (discourse & \multicolumn{2}{c}{\textbf{Production pressures}}\\
\centering accessibility, \small \citeNP{reed1991}) & \multicolumn{1}{c}{Availability} & \multicolumn{1}{c}{UID} \tabularnewline
\midrule
\red{- previous mention of NP} & \blue{- frequency of head} & \green{- I(\normalsize{SOME}\large |  NP head)} \\
\red{- topicality of \emph{some}-NP} & \blue{- animacy of head} & \green{- I(\normalsize{SOME}\large |  previous word)} \\
\red{- modification of head} & \multicolumn{2}{c}{\multirow{2}{*}{Alex $\underbrace{\textrm{ate}}_{\textrm{\green{previous word}}}$ some (of the) $\underbrace{\textrm{cashews}}_{\textrm{\green{NP head}}}$}} \\
\red{- head type (mass/count)} \\
\bottomrule
\end{tabular}

\vspace{.4cm}

I(\normalsize{SOME} \large | context) = $- \log (p(\textrm{\emph{some | }}  \textrm{context}) +  p(\textrm{\emph{some of DT | }}  \textrm{context}))$

\end{textblock}


\begin{textblock}{7.14}(7.6,13.95)
  \begin{center}
   \mbox{\colorbox{white}{
         \begin{minipage}{1.0\textwidth}
         \white{--------------------------------------------------------------------------}
	\vspace{12cm}
         \end{minipage}
      }
   }
\end{center}
\end{textblock}



\begin{textblock}{4.1}(7.6,14.4)
%\includegraphics[width=\textwidth]{pics/coefficients}
\end{textblock}

\begin{textblock}{3}(11.8,14.4)
\red{Meaning} factors are strongest, but both \green{UID} factors and one  \blue{availability} factor affect partitive choice in  predicted direction: more partitives with increasing information of \normalsize SOME \large and decreasing availability of head.
\end{textblock}

\begin{textblock}{7.1}(7.7,14.35)
\textbf{Results}
\end{textblock}

\end{textblock}


%%%%%%%%%%%%%%%%%%%%%%%%%%%%%%%%%%%%%%%
% TESTING HYPOTHESIS 2
%%%%%%%%%%%%%%%%%%%%%%%%%%%%%%%%%%%%%%%

\begin{textblock}{7.2}(-0.1,12.97)
  \begin{center}
   \mbox{\colorbox{superlightgray}{
         \begin{minipage}{1.0\textwidth}
         \superlightgray{--------------------------------------------------------------------------}
	\vspace{43.65cm}
         \end{minipage}
      }
   }
\end{center}
\end{textblock} 

\begin{textblock}{2.65}(7,17.7)
  \begin{center}
   \mbox{\colorbox{superlightgray}{
         \begin{minipage}{1.0\textwidth}
         \superlightgray{--------------------------------------------------------------------------}
	\vspace{23.95cm}
         \end{minipage}
      }
   }
\end{center}
\end{textblock} 

\begin{textblock}{7}(0,13.37)
  \LHead{Study 2: Gradient Alternation Hypothesis}
  
\begin{textblock}{6.8}(0.1,13.6)
  \begin{center}
   \mbox{\colorbox{white}{
         \begin{minipage}{1.0\textwidth}
         \white{--------------------------------------------------------------------------}
	\vspace{13.5cm}
         \end{minipage}
      }
   }
\end{center}
\end{textblock} 

\begin{textblock}{6.8}(0.2,13.9)
\large
%\includegraphics[width=.55\textwidth]{pics/weakstrong.png}

\vspace{-0.5cm}
2 meanings:  weak \emph{sm} vs. strong \emph{some}\\ \small \cite{milsark1974, ladusaw1994, horn1997} \large \\
\textbf{Methods.} For each case, collected 10 \\ similarity ratings of original to sentence with \emph{some (of)} omitted to obtain a measure of \emph{some}-NP strength (exploiting presuppositionality)

\end{textblock} 

%%%
\begin{textblock}{3}(4.05,13.95)
\large

\hspace{1.5cm} \textbf{Results} of MTurk ratings

%\includegraphics[width=0.96\textwidth]{pics/partitivedist}
\end{textblock} 


\begin{textblock}{6.8}(0.1,17.5)
  \begin{center}
   \mbox{\colorbox{white}{
         \begin{minipage}{1.0\textwidth}
         \white{--------------------------------------------------------------------------}
	\vspace{26cm}
         \end{minipage}
      }
   }
\end{center}
\end{textblock} 

\begin{textblock}{3}(6.45,18.5)
  \begin{center}
   \mbox{\colorbox{white}{
         \begin{minipage}{1.0\textwidth}
         \white{--------------------------------------------------------------------------}
	\vspace{21.2cm}
         \end{minipage}
      }
   }
\end{center}
\end{textblock} 

\begin{textblock}{6.6}(0.2,17.9)
\large

\textbf{Results} of fitting model to partitive and weak / strong simple \emph{some}

\end{textblock} 


\begin{textblock}{9.5}(0.1,18.2)
\large

%\includegraphics[width=\textwidth]{pics/gradientAlternation}
\end{textblock} 

\begin{textblock}{0.01}(6.5,18.595)
  \begin{center}
   \mbox{\colorbox{red}{
         \begin{minipage}{1.0\textwidth}
         \red{}
	\vspace{8.15cm}
         \end{minipage}
      }
   }
\end{center}
\end{textblock} 

\begin{textblock}{0.01}(6.5,21.2)
  \begin{center}
   \mbox{\colorbox{blue}{
         \begin{minipage}{1.0\textwidth}
         \blue{}
	\vspace{3.3cm}
         \end{minipage}
      }
   }
\end{center}
\end{textblock} 

\begin{textblock}{0.01}(6.5,22.51)
  \begin{center}
   \mbox{\colorbox{green}{
         \begin{minipage}{1.0\textwidth}
         \green{}
	\vspace{3.3cm}
         \end{minipage}
      }
   }
\end{center}
\end{textblock} 

\begin{textblock}{0.01}(0.63,18.115)
  \begin{center}
   \mbox{\colorbox{weakpink}{
         \begin{minipage}{1.0\textwidth}
         \weakpink{}
	\vspace{0.45cm}
         \end{minipage}
      }
   }
\end{center}
\end{textblock}

\begin{textblock}{0.01}(3.4,18.115)
  \begin{center}
   \mbox{\colorbox{weakpink}{
         \begin{minipage}{1.0\textwidth}
         \weakpink{}
	\vspace{19.6cm}
         \end{minipage}
      }
   }
\end{center}
\end{textblock}

\begin{textblock}{0.01}(3.56,18.115)
  \begin{center}
   \mbox{\colorbox{strongyellow}{
         \begin{minipage}{1.0\textwidth}
         \strongyellow{}
	\vspace{19.6cm}
         \end{minipage}
      }
   }
\end{center}
\end{textblock}

\begin{textblock}{0.01}(6.31,18.115)
  \begin{center}
   \mbox{\colorbox{strongyellow}{
         \begin{minipage}{1.0\textwidth}
         \strongyellow{}
	\vspace{0.45cm}
         \end{minipage}
      }
   }
\end{center}
\end{textblock}
 
\begin{textblock}{2.5}(6.7,18.3)
simple \emph{some} strength determined by similarity breakpoints

\end{textblock}  

\begin{textblock}{5.25}(9.7,23.95)
  \begin{center}
   \mbox{\colorbox{superlightgray}{
         \begin{minipage}{1.0\textwidth}
         \superlightgray{}
	\vspace{3.1cm}
         \end{minipage}
      }
   }
\end{center}
\end{textblock}

\begin{textblock}{5.25}(9.5,24.15)
  \begin{center}
   \mbox{\colorbox{white}{
         \begin{minipage}{1.0\textwidth}
         \white{}
	\vspace{1.5cm}
         \end{minipage}
      }
   }
\end{center}
\end{textblock}

\begin{textblock}{15}(0.2,24.5)
\large

meaning factors contribute to both weak/strong and simple/partitive difference
-- production pressures  primarily to simple/partitive (strong dataset)

\end{textblock} 

\end{textblock} 

%%%%%%%%%%%%%%%%%%%%%%%%%%%%%%%%%%%%%%%
% MEANING DIFFERENCES
%%%%%%%%%%%%%%%%%%%%%%%%%%%%%%%%%%%%%%%

%\begin{textblock}{7.5}(7.5,6.45)
%  \LHead{Assessing meaning differences}
%\large

%\vspace{0.5cm}

%\mbox{\colorbox{MyGray}{\begin{minipage}{\textwidth}\vspace{5pt}\emph{\white{Partitive Constraint}} \white{(\citeNP{jackendoff1977}; \citeNP{ladusaw1982})\large :\\ The embedded NP inside the \emph{of}-PP must be definite/specific.} \vspace{5pt}\end{minipage}}}

%\vspace{0.5cm}
%	
%  \begin{itemize}
%	\item \large givenness as predictor of definiteness in Switchboard corpus
%	\item \large 3-level givenness annotation for 25\% of Switchboard:  \textbf{new} (NP referent is new to discourse), \textbf{mediated} (NP referent is new to discourse but \\ can be inferred from context), \textbf{given} (NP referent has \\ been previously mentioned)
%  \end{itemize}
%\end{textblock} 

%\begin{textblock}{6.5}(8.1,6.95)
%	\begin{figure}
%		\includegraphics{pics/nps.png}
%	\end{figure}
%\end{textblock} 

%\begin{textblock}{2.5}(7.5,8.15)
%\centering
%%\textbf{All NPs}
%	\begin{figure}
%%		\caption{\textbf{All NPs}}	
%		\includegraphics[width=\textwidth]{pics/givennessbyDef}
%	\end{figure}
%\end{textblock} 	

%\begin{textblock}{2.5}(10,8.15)
%\centering
%%\textbf{\emph{some}-NPs}
%	\begin{figure}
%%		\caption{\textbf{\emph{some-}NPs}}	
%		\includegraphics[width=\textwidth]{pics/givennessbyPartPron}
%	\end{figure}
%\end{textblock} 	

%
%\begin{textblock}{2.5}(12.5,8.1)
%\centering
%%\textbf{\emph{some-}NPs (no pronouns)}
%	\begin{figure}
%%		\caption{\textbf{\emph{some-}NPs (no pronouns)}}
%		\includegraphics[width=\textwidth]{pics/givennessbyPart}
%	\end{figure}
%\end{textblock} 

%\begin{textblock}{7.5}(7.5,11.15)
%\large
%  Definite NPs and partitive \emph{some}-NPs are more likely with more given referents ($p < .0001$).
%\end{textblock}

%%%%%%%%%%%%%%%%%%%%%%%%%%%%%%%%%%%%%%%%
%% DATASET
%%%%%%%%%%%%%%%%%%%%%%%%%%%%%%%%%%%%%%%%

%\begin{textblock}{7.5}(7.5,12)
%  \LHead{The dataset}
%  \large

%\begin{itemize}
%	\item \large 1951 cases of \emph{some}-NPs from Switchboard corpus
%	\item \large excluded cases that could only occur in one of the \\two forms; if NP head was
%	\begin{itemize}
%		\item \large a pronoun (only partitive form)
%		\item \large a singular count noun (only simple form)
%	\end{itemize}
%	\item \large \textbf{total 1139 cases (233 partitives)}
%\end{itemize}

%\end{textblock}

%\begin{textblock}{4.52}(7.4,11.7)
%  \begin{center}
%   \mbox{\colorbox{superlightgray}{
%         \begin{minipage}{1.0\textwidth}
%         \superlightgray{--------------------------------------------------------------------------}
%	\vspace{12.9cm}
%         \end{minipage}
%      }
%   }
%\end{center}

%\end{textblock} 
%  
%%%%%%%%%%%%%%%%%%%%%%%%%%%%%%%%%%%%%%%%
%% ROBUST COMMUNICATION
%%%%%%%%%%%%%%%%%%%%%%%%%%%%%%%%%%%%%%%%
%  

%\begin{textblock}{2.8}(12.35,12)
%  \LHead{Quantifying robust \\communication}

%\large

%   Alex \red{ate} $[_{\textrm{\green{OBJ}}}$ \textsc{some} \blue{cashews} $]$ 
%  
%  \begin{itemize}
%		\item \green{Info(\normalsize{SOME} \large | NP grammatical \\function)}  
%		\item \blue{Info(\normalsize{SOME} \large |  NP head)}		
%		\item \red{Info(\normalsize{SOME} \large | previous word)}
%  \end{itemize}
%  

%\end{textblock}  

%
%%\begin{textblock}{2.6}(12.27,11.72)
%%  \begin{center}
%%   \mbox{\colorbox{white}{
%%         \begin{minipage}{1.0\textwidth}
%%         \white{--------------------------------------------------------------------------}
%%	\vspace{11.65cm}
%%         \end{minipage}
%%      }
%%   }
%%\end{center}

%%\end{textblock} 

%%\begin{textblock}{2.64}(12.25,11.7)
%%  \begin{center}
%%   \mbox{\colorbox{black}{
%%         \begin{minipage}{1.0\textwidth}
%%         \white{--------------------------------------------------------------------------}
%%	\vspace{11.85cm}
%%         \end{minipage}
%%      }
%%   }
%%\end{center}

%%\end{textblock} 

%%%%%%%%%%%%%%%%%%%%%%%%%%%%%%%%%%%%%%%%
%% MODEL
%%%%%%%%%%%%%%%%%%%%%%%%%%%%%%%%%%%%%%%%

%%\begin{textblock}{7.5}(7.5,11.2)
%%  \LHead{The model}

%%\large
%%\begin{itemize}
%%	\item mixed effects logistic regression predicting partitive vs.~simple form
%%	\item fixed effects of interest: 3 informational variables
%%	\item meaning control: givenness (categorical variable: \verb|new| or \verb|not new|)
%%	\item availability controls: predictability and animacy (\verb|inanimate| or \verb|animate|) of NP head, whether head is mass or count noun
%%	\item random effects: random speaker intercept
%%\end{itemize}
%%\end{textblock}

%%%%%%%%%%%%%%%%%%%%%%%%%%%%%%%%%%%%%%%%
%% RESULTS
%%%%%%%%%%%%%%%%%%%%%%%%%%%%%%%%%%%%%%%%

%\begin{textblock}{5.5}(9.5,15.1)
%  \LHead{Results}

%\large
%\begin{itemize}
%	\item mixed effects logistic regression predicting partitive vs.~simple \normalsize SOME \large (random speaker intercept)
%	\item \large \textbf{partitive more likely as information of \normalsize SOME \large increases} %($\beta$=0.11, 0.16, 0.45, \\$SE$=0.04, 0.05, 0.11, $p<$.01, .01, .001)
%	\item \large partitive more likely with given heads %($\beta$=0.4, $SE$=0.16, $p<$.05)
%	\item \large robust communication measures contribute to the choice of partitive vs.~simple \normalsize SOME \large above and beyond meaning differences as captured by givenness 
%	\item control predictors: 
%	\begin{itemize}
%		\item no effect of head predictability %($\beta$=0.06, $SE$=0.04, $p$=.13)
%		\item partitive more likely with animate heads % ($\beta$=0.45 , $SE$=0.22, $p<$.05)
%		\item partitive more likely with count nouns (largest effect) % ($\beta$= 1.59, $SE$=0.22, $p<$.0001)
%	\end{itemize}	
%\end{itemize}

%\end{textblock}

%\begin{textblock}{5.5}(9.45,14.9)
%  \begin{center}
%   \mbox{\colorbox{xxlightgray}{
%         \begin{minipage}{1.0\textwidth}
%         \xxlightgray{--------------------------------------------------------------------------}
%	\vspace{19cm}
%         \end{minipage}
%      }
%   }
%\end{center}

%\end{textblock} 
%  
%\begin{textblock}{3}(6.2,15.5)
%	\begin{figure}
%		\caption{\textbf{Bigram predictability}}	
%		\includegraphics[width=\textwidth]{pics/infopre0.png}
%	\end{figure}
%\end{textblock} 	

%\begin{textblock}{3}(0,15.5)
%	\begin{figure}
%		\caption{\textbf{Grammatical function}}	
%%		\includegraphics[width=\textwidth]{pics/infogf0.png}
%		\includegraphics[width=\textwidth]{pics/infogf0withGF.png}
%	\end{figure}
%\end{textblock} 

%\begin{textblock}{3}(3.1,15.5)
%	\begin{figure}
%		\caption{\textbf{NP head}}	
%		\includegraphics[width=\textwidth]{pics/infohead0.png}
%	\end{figure}
%\end{textblock}

%\begin{textblock}{2}(13.3,19.65)
%\normalsize
%Total Nagelkerke R$^2$=0.17
%\end{textblock}

%\begin{textblock}{1}(6.2,20.6)
%\begin{sideways}
%\textbf{Predictor}
%\end{sideways}
%\end{textblock}

%\begin{textblock}{9.2}(6,19.4)
%%	\begin{figure}
%		\includegraphics[width=\textwidth]{pics/twoplots.png}
%%	\end{figure}
%\end{textblock}





%%%%%%%%%%%%%%%%%%%%%%%%%%%%%%%%%%%%%%%
% CONCLUSION
%%%%%%%%%%%%%%%%%%%%%%%%%%%%%%%%%%%%%%%

  \begin{textblock}{4.95}(10,17.7)
  \begin{center}
   \mbox{\colorbox{superlightgray}{
         \begin{minipage}{1.0\textwidth}
         \superlightgray{--------------------------------------------------------------------------}
	\vspace{21cm}
         \end{minipage}
      }
   }
\end{center}

\end{textblock} 

\begin{textblock}{4.45}(10.2,18.1)
  \LHead{Conclusion}

\large
	\begin{itemize}
		\item \large \textbf{Production pressures apply to the choice between forms that are not meaning - \\equivalent}. We conclude that, rather than production pressures applying after restriction of permissible forms by semantics, \textbf{the pressure to robustly communicate a core meaning applies in parallel with the pressure to find the most precise form} to encode an intended meaning.
		\item \large  This is compatible with probabilistic approaches to form choice: the more similar the meaning (and associated inferences) of two forms, the more likely the choice between the two is to be affected by production pressures.
%		\item \Large Several control predictors compatible with Availability-based production (\citeNP{ferreira2000}) have effect in unexpected direction. Possibly due to inherent correlation of controls with meaning distinction between the two forms (specificity or individuatability).
	\end{itemize}

\end{textblock} 



%%%%%%%%%%%%%%%%%%%%%%%%%%%%%%%%%%%%%%%
% REFERENCES
%%%%%%%%%%%%%%%%%%%%%%%%%%%%%%%%%%%%%%%
\renewcommand{\refname}{\normalsize References}

\begin{textblock}{15}(0,26)
\small
\textbf{References}
\end{textblock}

\renewcommand{\refname}{}
\begin{textblock}{15}(0,25)
\let\bibliographytypesize\scriptsize\bibliographystyle{apacitex}
\bibliographystyle{apacitex}
\bibliography{bibs}  
\end{textblock}

%\begin{textblock}{8.66}(6.44,22.9)
%  \begin{center}
%   \mbox{\colorbox{white}{
%         \begin{minipage}{1.0\textwidth}
%         \white{--------------------------------------------------------------------------}
%	\vspace{8.5cm}
%         \end{minipage}
%      }
%   }
%\end{center}

%\end{textblock} 

%\begin{textblock}{8.7}(6.42,22.88)
%  \begin{center}
%   \mbox{\colorbox{MyGray}{
%         \begin{minipage}{1.0\textwidth}
%         \white{--------------------------------------------------------------------------}
%	\vspace{8.7cm}
%         \end{minipage}
%      }
%   }
%\end{center}

%\end{textblock} 

\end{document}

